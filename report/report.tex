\PassOptionsToPackage{unicode=true}{hyperref} % options for packages loaded elsewhere
\PassOptionsToPackage{hyphens}{url}
%
\documentclass[]{article}
\usepackage{lmodern}
\usepackage{amssymb,amsmath}
\usepackage{ifxetex,ifluatex}
\usepackage{fixltx2e} % provides \textsubscript
\ifnum 0\ifxetex 1\fi\ifluatex 1\fi=0 % if pdftex
  \usepackage[T1]{fontenc}
  \usepackage[utf8]{inputenc}
  \usepackage{textcomp} % provides euro and other symbols
\else % if luatex or xelatex
  \usepackage{unicode-math}
  \defaultfontfeatures{Ligatures=TeX,Scale=MatchLowercase}
\fi
% use upquote if available, for straight quotes in verbatim environments
\IfFileExists{upquote.sty}{\usepackage{upquote}}{}
% use microtype if available
\IfFileExists{microtype.sty}{%
\usepackage[]{microtype}
\UseMicrotypeSet[protrusion]{basicmath} % disable protrusion for tt fonts
}{}
\IfFileExists{parskip.sty}{%
\usepackage{parskip}
}{% else
\setlength{\parindent}{0pt}
\setlength{\parskip}{6pt plus 2pt minus 1pt}
}
\usepackage{hyperref}
\hypersetup{
            pdftitle={Stat 363: Final Project},
            pdfauthor={Authors: Woods Connell, Hannah Knight, Charles Wong, and Samuel Helms},
            pdfborder={0 0 0},
            breaklinks=true}
\urlstyle{same}  % don't use monospace font for urls
\usepackage{longtable,booktabs}
% Fix footnotes in tables (requires footnote package)
\IfFileExists{footnote.sty}{\usepackage{footnote}\makesavenoteenv{longtable}}{}
\usepackage{graphicx,grffile}
\makeatletter
\def\maxwidth{\ifdim\Gin@nat@width>\linewidth\linewidth\else\Gin@nat@width\fi}
\def\maxheight{\ifdim\Gin@nat@height>\textheight\textheight\else\Gin@nat@height\fi}
\makeatother
% Scale images if necessary, so that they will not overflow the page
% margins by default, and it is still possible to overwrite the defaults
% using explicit options in \includegraphics[width, height, ...]{}
\setkeys{Gin}{width=\maxwidth,height=\maxheight,keepaspectratio}
\setlength{\emergencystretch}{3em}  % prevent overfull lines
\providecommand{\tightlist}{%
  \setlength{\itemsep}{0pt}\setlength{\parskip}{0pt}}
\setcounter{secnumdepth}{0}
% Redefines (sub)paragraphs to behave more like sections
\ifx\paragraph\undefined\else
\let\oldparagraph\paragraph
\renewcommand{\paragraph}[1]{\oldparagraph{#1}\mbox{}}
\fi
\ifx\subparagraph\undefined\else
\let\oldsubparagraph\subparagraph
\renewcommand{\subparagraph}[1]{\oldsubparagraph{#1}\mbox{}}
\fi

% set default figure placement to htbp
\makeatletter
\def\fps@figure{htbp}
\makeatother


\title{Stat 363: Final Project}
\author{Authors: Woods Connell, Hannah Knight, Charles Wong, and Samuel Helms}
\date{}

\begin{document}
\maketitle

{
\setcounter{tocdepth}{3}
\tableofcontents
}
\hypertarget{introduction}{%
\section{Introduction}\label{introduction}}

It is extremely common to look at statistics like points scored and
rebounds grabbed when evaluating a basketball player. As much as these
statistics can tell us about a player, what a player does when he is off
the ball has the potential to say even more: after all, a player only
has a ball for a fraction of the game. In this report, we use data on
NBA player's court location, which we sample at a rate of every half
second of over 300 games from the 2016 season.

\hypertarget{design-and-primary-questions}{%
\section{Design and Primary
Questions}\label{design-and-primary-questions}}

We set out to answer two questions about the movement data and NBA
players/teams in this paper:

\begin{enumerate}
\def\labelenumi{\arabic{enumi}.}
\item
  Can we can detect different positions based on where players stand on
  the court? We use factor analysis and PCA on a matrix that has each
  player as a data point and fourteen court locations as features, and
  also on a matrix that has each team as a data point and the fourteen
  court locations as features.
\item
  Do some teams, like the warriors, spend more time out by the three
  point line? Do others crowd the basket? We try to detect what effect,
  if any, being on a team has on the locations players inhabit on the
  court. We use Multivariate Analysis of Variance and Linear
  Discriminant Analysis to answer these questions.
\end{enumerate}

\hypertarget{data}{%
\section{Data}\label{data}}

Movement data is no longer released by the NBA, and so it was collected
from a public github repository
(\url{https://github.com/sealneaward/nba-movement-data}) that backs the
data up. In its raw form, this dataset is large in scale, on the scale
of 50 gigabytes. It consists of x and y coordinates for every player on
the court and the basketball, sampled on .02 second intervals. The NBA
only released data for the 2016 season. Our dataset is comprised of
about 300 games from the course of that season.

All of our code for data preparation is hosted in the following git
repository: https://github.com/samghelms/nba\_movement\_analysis.
Specifically, you can find the data used for this report in the
\texttt{data} folder. You can build the datasets used for this analysis
by running \texttt{make\_data.py}. You can walk through the code to
create the plots in the\texttt{pca\ and\ factor\ analysis} jupyter
notebook in our github repository - (link)
\url{https://github.com/samghelms/nba_movement_analysis/blob/master/pca\%20and\%20factor\%20analysis.ipynb}.

To winnow the data down, we decided to extract player locations on half
second intervals. After doing that, we needed to engineer some features
from the player locations: The raw movement data does not come with zone
labels. To get these labels, we used a k-nearest neighbors classifier on
a dataset of labeled shots. We used code from the following repository
to create these labels:
\url{https://github.com/sealneaward/movement-quadrants}. The author of
this repository created the labels by using labeleld shot logs from Kobe
Bryant's NBA career.

The table below summarizes our features. We apply various other grouping
and striding operations to the data shown below, but this is the set of
engineered features that the data sources for all the following analyses
stem from.

\hypertarget{the-features}{%
\subsection{The features}\label{the-features}}

\begin{longtable}[]{@{}lll@{}}
\toprule
\begin{minipage}[b]{0.32\columnwidth}\raggedright
Feature\strut
\end{minipage} & \begin{minipage}[b]{0.12\columnwidth}\raggedright
Data Type\strut
\end{minipage} & \begin{minipage}[b]{0.47\columnwidth}\raggedright
Description\strut
\end{minipage}\tabularnewline
\midrule
\endhead
\begin{minipage}[t]{0.32\columnwidth}\raggedright
\strut
\end{minipage} & \begin{minipage}[t]{0.12\columnwidth}\raggedright
\strut
\end{minipage} & \begin{minipage}[t]{0.47\columnwidth}\raggedright
\strut
\end{minipage}\tabularnewline
\begin{minipage}[t]{0.32\columnwidth}\raggedright
Offense or Defense\strut
\end{minipage} & \begin{minipage}[t]{0.12\columnwidth}\raggedright
Binary\strut
\end{minipage} & \begin{minipage}[t]{0.47\columnwidth}\raggedright
\strut
\end{minipage}\tabularnewline
\begin{minipage}[t]{0.32\columnwidth}\raggedright
Player name\strut
\end{minipage} & \begin{minipage}[t]{0.12\columnwidth}\raggedright
Categorical\strut
\end{minipage} & \begin{minipage}[t]{0.47\columnwidth}\raggedright
\strut
\end{minipage}\tabularnewline
\begin{minipage}[t]{0.32\columnwidth}\raggedright
Team\strut
\end{minipage} & \begin{minipage}[t]{0.12\columnwidth}\raggedright
Categorical\strut
\end{minipage} & \begin{minipage}[t]{0.47\columnwidth}\raggedright
\strut
\end{minipage}\tabularnewline
\begin{minipage}[t]{0.32\columnwidth}\raggedright
Right Side(R), 8-16 ft.\strut
\end{minipage} & \begin{minipage}[t]{0.12\columnwidth}\raggedright
Continuous\strut
\end{minipage} & \begin{minipage}[t]{0.47\columnwidth}\raggedright
Count data (number of half second intervals in)\strut
\end{minipage}\tabularnewline
\begin{minipage}[t]{0.32\columnwidth}\raggedright
Right Side(R), 16-24 ft.\strut
\end{minipage} & \begin{minipage}[t]{0.12\columnwidth}\raggedright
Continuous\strut
\end{minipage} & \begin{minipage}[t]{0.47\columnwidth}\raggedright
\strut
\end{minipage}\tabularnewline
\begin{minipage}[t]{0.32\columnwidth}\raggedright
Right Side(R), 24+ ft.\strut
\end{minipage} & \begin{minipage}[t]{0.12\columnwidth}\raggedright
Continuous\strut
\end{minipage} & \begin{minipage}[t]{0.47\columnwidth}\raggedright
\strut
\end{minipage}\tabularnewline
\begin{minipage}[t]{0.32\columnwidth}\raggedright
Left Side(L), 8-16 ft.\strut
\end{minipage} & \begin{minipage}[t]{0.12\columnwidth}\raggedright
Continuous\strut
\end{minipage} & \begin{minipage}[t]{0.47\columnwidth}\raggedright
\strut
\end{minipage}\tabularnewline
\begin{minipage}[t]{0.32\columnwidth}\raggedright
Left Side(L), 16-24 ft.\strut
\end{minipage} & \begin{minipage}[t]{0.12\columnwidth}\raggedright
Continuous\strut
\end{minipage} & \begin{minipage}[t]{0.47\columnwidth}\raggedright
\strut
\end{minipage}\tabularnewline
\begin{minipage}[t]{0.32\columnwidth}\raggedright
Left Side(L), 24+ ft.\strut
\end{minipage} & \begin{minipage}[t]{0.12\columnwidth}\raggedright
Continuous\strut
\end{minipage} & \begin{minipage}[t]{0.47\columnwidth}\raggedright
\strut
\end{minipage}\tabularnewline
\begin{minipage}[t]{0.32\columnwidth}\raggedright
Center(C), Less Than 8 ft.\strut
\end{minipage} & \begin{minipage}[t]{0.12\columnwidth}\raggedright
Continuous\strut
\end{minipage} & \begin{minipage}[t]{0.47\columnwidth}\raggedright
\strut
\end{minipage}\tabularnewline
\begin{minipage}[t]{0.32\columnwidth}\raggedright
Center(C), 8-16 ft.\strut
\end{minipage} & \begin{minipage}[t]{0.12\columnwidth}\raggedright
Continuous\strut
\end{minipage} & \begin{minipage}[t]{0.47\columnwidth}\raggedright
\strut
\end{minipage}\tabularnewline
\begin{minipage}[t]{0.32\columnwidth}\raggedright
Center(C), 16-24 ft.\strut
\end{minipage} & \begin{minipage}[t]{0.12\columnwidth}\raggedright
Continuous\strut
\end{minipage} & \begin{minipage}[t]{0.47\columnwidth}\raggedright
\strut
\end{minipage}\tabularnewline
\begin{minipage}[t]{0.32\columnwidth}\raggedright
Center(C), 24+ ft.\strut
\end{minipage} & \begin{minipage}[t]{0.12\columnwidth}\raggedright
Continuous\strut
\end{minipage} & \begin{minipage}[t]{0.47\columnwidth}\raggedright
\strut
\end{minipage}\tabularnewline
\begin{minipage}[t]{0.32\columnwidth}\raggedright
Right Side Center(RC), 16-24 ft.\strut
\end{minipage} & \begin{minipage}[t]{0.12\columnwidth}\raggedright
Continuous\strut
\end{minipage} & \begin{minipage}[t]{0.47\columnwidth}\raggedright
\strut
\end{minipage}\tabularnewline
\begin{minipage}[t]{0.32\columnwidth}\raggedright
Right Side Center(RC), 24+ ft.\strut
\end{minipage} & \begin{minipage}[t]{0.12\columnwidth}\raggedright
Continuous\strut
\end{minipage} & \begin{minipage}[t]{0.47\columnwidth}\raggedright
\strut
\end{minipage}\tabularnewline
\begin{minipage}[t]{0.32\columnwidth}\raggedright
Left Side Center(LC), 16-24 ft.\strut
\end{minipage} & \begin{minipage}[t]{0.12\columnwidth}\raggedright
Continuous\strut
\end{minipage} & \begin{minipage}[t]{0.47\columnwidth}\raggedright
\strut
\end{minipage}\tabularnewline
\begin{minipage}[t]{0.32\columnwidth}\raggedright
Left Side Center(LC), 24+ ft.\strut
\end{minipage} & \begin{minipage}[t]{0.12\columnwidth}\raggedright
Continuous\strut
\end{minipage} & \begin{minipage}[t]{0.47\columnwidth}\raggedright
\strut
\end{minipage}\tabularnewline
\bottomrule
\end{longtable}

The data is very complete--there aren't many skipped half seconds--so we
do not have many concerns about error via omission. The biggest
potential error would come from our half-second sampling methodology: in
a half second, a fast-moving player might cover the space of half the
court. We hope that this error will average out in the long run,
however.

\hypertarget{descriptive-plots-summary-statistics}{%
\section{Descriptive plots, summary
statistics}\label{descriptive-plots-summary-statistics}}

\hypertarget{three-subsets-of-the-data}{%
\subsection{Three subsets of the data}\label{three-subsets-of-the-data}}

We aggregate the data to several levels for the different analyses used
in this report. The results of all three of the aggregations for the
location data look solid: all have normal distribution of court
locations and somewhat linear relationships between all pairs of court
locations. The following sections will go into further detail.

\hypertarget{first-subset}{%
\subsubsection{First subset}\label{first-subset}}

The first way in which we aggregate the data is to group by team and
offense/defense status and count the number of half seconds players on
each team are in specific court locations. We don't want to lump offense
and defense in with each other, since standing in a court location on
defense can have a very different meaning than standing in the same
location on offense.

This data is available at \{INS\_LINK\}

The first few data points in the dataset, for teams on offense, look
like this (some features, like `Right Side Center(RC), 16-24 ft.' have
been omitted for clarity; there are actually 14 court locations that we
look at):

\begin{longtable}[]{@{}llllll@{}}
\toprule
\begin{minipage}[b]{0.25\columnwidth}\raggedright
Court location\strut
\end{minipage} & \begin{minipage}[b]{0.12\columnwidth}\raggedright
Atlanta Hawks.\strut
\end{minipage} & \begin{minipage}[b]{0.12\columnwidth}\raggedright
Boston Celtics\strut
\end{minipage} & \begin{minipage}[b]{0.10\columnwidth}\raggedright
Cleveland Cavaliers\strut
\end{minipage} & \begin{minipage}[b]{0.12\columnwidth}\raggedright
New Orleans Pelicans\strut
\end{minipage} & \begin{minipage}[b]{0.12\columnwidth}\raggedright
Chicago Bulls\strut
\end{minipage}\tabularnewline
\midrule
\endhead
\begin{minipage}[t]{0.25\columnwidth}\raggedright
Right Side(R), 16-24 ft.\strut
\end{minipage} & \begin{minipage}[t]{0.12\columnwidth}\raggedright
39441.0\strut
\end{minipage} & \begin{minipage}[t]{0.12\columnwidth}\raggedright
36185.0\strut
\end{minipage} & \begin{minipage}[t]{0.10\columnwidth}\raggedright
33619.0\strut
\end{minipage} & \begin{minipage}[t]{0.12\columnwidth}\raggedright
37517.0\strut
\end{minipage} & \begin{minipage}[t]{0.12\columnwidth}\raggedright
33116.0\strut
\end{minipage}\tabularnewline
\begin{minipage}[t]{0.25\columnwidth}\raggedright
Left Side(L), 24+ ft.\strut
\end{minipage} & \begin{minipage}[t]{0.12\columnwidth}\raggedright
30794.0\strut
\end{minipage} & \begin{minipage}[t]{0.12\columnwidth}\raggedright
32525.0\strut
\end{minipage} & \begin{minipage}[t]{0.10\columnwidth}\raggedright
27419.0\strut
\end{minipage} & \begin{minipage}[t]{0.12\columnwidth}\raggedright
30732.0\strut
\end{minipage} & \begin{minipage}[t]{0.12\columnwidth}\raggedright
28469.0\strut
\end{minipage}\tabularnewline
\begin{minipage}[t]{0.25\columnwidth}\raggedright
Center(C), Less Than 8 ft.\strut
\end{minipage} & \begin{minipage}[t]{0.12\columnwidth}\raggedright
178075.0\strut
\end{minipage} & \begin{minipage}[t]{0.12\columnwidth}\raggedright
165085.0\strut
\end{minipage} & \begin{minipage}[t]{0.10\columnwidth}\raggedright
132478.0\strut
\end{minipage} & \begin{minipage}[t]{0.12\columnwidth}\raggedright
152651.0\strut
\end{minipage} & \begin{minipage}[t]{0.12\columnwidth}\raggedright
152355.0\strut
\end{minipage}\tabularnewline
\begin{minipage}[t]{0.25\columnwidth}\raggedright
Right Side Center(RC), 24+ ft.\strut
\end{minipage} & \begin{minipage}[t]{0.12\columnwidth}\raggedright
126667.0\strut
\end{minipage} & \begin{minipage}[t]{0.12\columnwidth}\raggedright
130071.0\strut
\end{minipage} & \begin{minipage}[t]{0.10\columnwidth}\raggedright
112729.0\strut
\end{minipage} & \begin{minipage}[t]{0.12\columnwidth}\raggedright
128496.0\strut
\end{minipage} & \begin{minipage}[t]{0.12\columnwidth}\raggedright
110340.0\strut
\end{minipage}\tabularnewline
\begin{minipage}[t]{0.25\columnwidth}\raggedright
Left Side Center(LC), 24+ ft.\strut
\end{minipage} & \begin{minipage}[t]{0.12\columnwidth}\raggedright
144596.0\strut
\end{minipage} & \begin{minipage}[t]{0.12\columnwidth}\raggedright
139255.0\strut
\end{minipage} & \begin{minipage}[t]{0.10\columnwidth}\raggedright
120485.0\strut
\end{minipage} & \begin{minipage}[t]{0.12\columnwidth}\raggedright
138640.0\strut
\end{minipage} & \begin{minipage}[t]{0.12\columnwidth}\raggedright
123351.0\strut
\end{minipage}\tabularnewline
\bottomrule
\end{longtable}

We use this data for factor analysis. We also keep track of what team
each player is on for use in the MANOVA later on.

\hypertarget{second-subset}{%
\subsubsection{Second subset}\label{second-subset}}

The second way in which we aggregate the data is to group by player and
count the number of half seconds each player is within a specific court
location. We use this data for Factor Analysis and also for MANOVA. The
layout of the data is similar to the table shown in the section above,
but is on the level of players, rather than teams. This data used for
the factor analysis is available at \{INS\_LINK\}, and the data used for
the MANOVA at \{INS\_LINK\}.

The following pair plot, on a subset of our player data for defense,
looks at how well these data fit our assumptions of multivariate
normality. You can see from the histograms and the scatter plots that
the data are normally distributed and have somewhat linear relationships
with each other. This makes us suspect that the data are a good fit for
multivariate analyses like MANOVA and linear discriminant analysis.

\begin{figure}
\centering
\includegraphics[width=0.8\textwidth,height=\textheight]{player_dist_and_relationships.png}
\caption{Relationships between court locations, and their distributions
for players on offense}
\end{figure}

\hypertarget{multivariate-analysis-and-discussion-of-results}{%
\section{Multivariate Analysis and discussion of
results}\label{multivariate-analysis-and-discussion-of-results}}

\hypertarget{factor-analysis-and-pca-distinguishing-players-and-teams-based-on-court-locations}{%
\subsection{Factor analysis and PCA: distinguishing players and teams
based on court
locations}\label{factor-analysis-and-pca-distinguishing-players-and-teams-based-on-court-locations}}

First, we will use factor analysis to try and identify court locations
that differentiate teams and players from one another. The results that
made the most sense came from running factor analysis on a matrix that
had been normalized across rows, so that each value corresponding to one
of the position parameters represented the probability of a player/team
being in that position, instead of being a raw count. We used Sci-Kit
Learn's Factor Analysis implementation, since we could only create the
plots used below in python (we had to develop a custom plot ourselves
using python's matplotlib package).

\hypertarget{players}{%
\subsubsection{Players}\label{players}}

First, we ran a PCA to decide how many loadings to use (for the matrices
of counts for offense and defense separately). A simple elbow plot
suggested that most of of the variance could be explained by the first
two loadings for both offense and defense. Because there are many unique
play-styles in the NBA, we decided to go ahead and use four components
-- explaining the style of a couple players with really unique styles in
our model isn't a bad thing.

\begin{figure}
\centering
\includegraphics{cum_pca_offense.png}
\caption{Cumulative explained variance, PCA, on offense data}
\end{figure}

Then, we used Factor Analysis on a correlation matrix of the data. We
chose a correlation matrix because there is such a playtime difference
between players, and we didn't want this to bias our data. We don't have
to worry about accidentally biasing our data because we have plenty of
data on a player even if they spend only minutes on the court.

The results are plotted below. Each element in the loading vector
corresponds to a court location, so we have plotted this value on an
actual court. Larger values (red in the plot) in a loading vector mean
that players that have a higher score for this factor will tend to be in
the corresponding area a lot, and smaller values (blue) indicate that a
player has a tendency to not be in those regions. We think of these
loading values as characterizing the play style of players: For example,
a larger value in the loading vector element corresponding with ``Right
Side(R), 8-16 ft.'' means that this the play style corresponding with
this loading characterized by playing a lot in this zone.

\hypertarget{offense}{%
\paragraph{Offense}\label{offense}}

First, we look at what locations distinguish players from each other on
offense. The factor analysis seems to suggest that there are two play
styles that characterize basketball players in the NBA:

\begin{figure}
\centering
\includegraphics[width=0.8\textwidth,height=\textheight]{first_4_loadings_players_off.png}
\caption{Plot of loadings for players on offense}
\end{figure}

\begin{itemize}
\item
  The first loading, is a player who stays in areas around the
  perimeter. These are likely players like guards and small forwards.
\item
  The second loading likely represents players like centers, who
  penetrate the center of the court and attack the rim: It is
  characterized by large values for zones close to the basket.
\item
  The third loading seems similar to the first, with the addition of the
  center. It is probably detecting guards and avoiding the noise
  surrounding many people running through the middle of the court
\item
  The fourth loading seems similar to the second, with the addition of
  perimeter shooting. It could be telling us that there are players who
  play in the middle and shoot from the perimeter, even though many of
  these players, like Joel Embiid, do exist.
\end{itemize}

It is very interesting to see that perimeter players are not
characterized by being in the center back-court. This is likely because
so players of all types pass through the center to get from one side of
the court to another, not because perimeter players never spend time in
the center of the back-court.

\hypertarget{defense}{%
\paragraph{Defense}\label{defense}}

Next, we look at what locations distinguish players from each other on
defense. A plot of cumulative variance explained for the PCA again
suggests that there will be up to 6 significant loadings.

The loadings for the factor analysis on defense are much less repetitive
than those on offense. All of the first four loadings have some unique
characteristics.

\begin{figure}
\centering
\includegraphics[width=0.8\textwidth,height=\textheight]{first_4_loadings_players_def.png}
\caption{Plot of loadings for players on defense}
\end{figure}

\begin{itemize}
\item
  The first loading appears to correspond with players who defend the
  sides of the court. It suggests that such players are generally
  ambidextrous in their ability to defend sides of the court. It is
  interesting to note how such players will often defend all the way up
  to the net against players on the far right or left. This is likely
  because players are less concentrated in these areas, and thus one man
  needs to cover more area on the right or left.
\item
  The second loading likely corresponds with players like center who
  defend the rim and spend a lot of time underneath the basket.
\item
  The third loading, if it is to be trusted, seems to correspond with a
  free-ranging type of player on defense, with scatterings of hot spots
  all over the court.
\item
  The fourth loading seems to be a player heavily focused on defending
  the perimeter, especially the far right and left regions of the court.
  It is believable that only certain types of players are likely to
  defend these regions--a taller player would not want to be drawn so
  far away from the basket, so it likely falls upon the smaller players
  to defend these far-drawn regions of the court.
\end{itemize}

\hypertarget{teams}{%
\subsubsection{Teams}\label{teams}}

\hypertarget{offense-1}{%
\paragraph{Offense}\label{offense-1}}

\begin{figure}
\centering
\includegraphics[width=0.8\textwidth,height=\textheight]{first_4_loadings_teams_off.png}
\caption{Plot of loadings for teams on offense}
\end{figure}

We perform the same analysis as above, just on a matrix with teams,
rather than players, as data points. This analysis is likely less
stable, since there are only 30 or so teams in the NBA, and we use
fourteen court locations, so the number of features is closer to the
number of data than ideal. In contrast with the PCA performed on the
players, the cumulative explained variance plot for teams does not reach
90\% until the fourth loading.

For offense, we see that some teams are more characterized by their
internal play, which makes up the first loading, while others are more
by their perimeter play, indicated by the the third loading. The team
play styles characterized the the second and fourth loading plots seem
to be offenses that are fairly evenly balanced across zones.

It is interesting to see how much more important interior play is on the
team level -- the first two loadings are heavily comprised of it. It
also makes sense to see perimeter play highlighted in the third, rather
than the first or second loadings, since only a few teams in the NBA,
like the Houston Rockets and Golden State Warriors, are heavily reliant
on their perimeter game, and so a relatively smaller amount of the
variance in the data is explained by playing in these zones.

\hypertarget{defense-1}{%
\paragraph{Defense}\label{defense-1}}

\begin{figure}
\centering
\includegraphics[width=0.8\textwidth,height=\textheight]{first_4_loadings_teams_def.png}
\caption{Plot of loadings for teams on defense}
\end{figure}

The defensive loadings show that some teams rely heavily on their
interior defense close to the net, while others (the second loading)
have a much more spread out defensive scheme. Loading three is
especially interesting, indicating that some teams almost never play
defense directly under the net. This could be teams like the Warriors,
who don't have an established big man to protect their rim, but do have
plenty of players in the 6'6 - 6'9 range who can play around the rim. It
also might just indicate that some teams don't like to risk jumping in
the way of some driving to the rim for a dunk, which risks a high-speed
midair collision.

\hypertarget{manova-and-lda-examining-the-effect-team-has-on-court-location-for-players.}{%
\subsection{MANOVA and LDA: Examining the effect team has on court
location for
players.}\label{manova-and-lda-examining-the-effect-team-has-on-court-location-for-players.}}

People often say that NBA teams like the Spurs have distinctive play
styles--``systems''. This section will try and verify this claim using
Multivariate Analysis of Variance and Linear Discriminant Analysis to
search for a relationship between team and the locations players stand
in on the court.

We find that there is not sufficient statistical evidence to suggest
that the group means of teams are different based on the F statistic of
the MANOVA test when run on offense and defense separately. However,
when run on the whole dataset, there is sufficient statistical evidence
to suggest that the group means of teams are different (with an F
statistic of 2.2e-16). This could be picking up on the differences
between offensive and defensive playing styles of different teams and
not locations or it could just be random noise. Therefore, we can
conclude that team does not play a large role in determining what zones
people spend time in.

\begin{longtable}[]{@{}llll@{}}
\toprule
Description & Wilks & FStatistic & Num\_Observations\tabularnewline
\midrule
\endhead
Defense & 0.268 & 0.669 & 322\tabularnewline
Offense & 0.272 & 0.766 & 319\tabularnewline
Both & 0.299 & 2.20E-16 & 641\tabularnewline
\bottomrule
\end{longtable}

You can read through the code we used to run LDA and generate our plots
at the following link:
\url{https://github.com/samghelms/nba_movement_analysis/blob/master/lda\%20analysis.ipynb}.

Trying to see if the LDA could show us anything at all, we plotted a
subset of the teams that we knew fairly well and plotted the data points
against the first two linear discriminants and labeled them with the
name of the player they represent, to see whether certain players, like
starters, were being clustered. The results of the LDA still appear to
be inconclusive, but we did see some evidence of clustering based on the
discriminants.

The one sign that the LDA model might be learning something detecting
some latent structure is that the Golden State Warriors are somewhat
clustered together relative to the other teams on the plot. This would
make sense, given that 2016 was the year that the warriors set the
league record for the amount of games won in a season.

\begin{figure}
\centering
\includegraphics{LDA_players_teams_offense.png}
\caption{Plot of first two discriminants for LDA on offense data}
\end{figure}

We saw even more clustering with the defensive data. This makes sense
since teams are more likely to have a set defensive scheme, and play a
more free-flowing offense. This makes the discriminants from the
offensive plot more credible: the Golden State Warriors were known to
have a unique offense in 2016.

\begin{figure}
\centering
\includegraphics{LDA_players_teams_defense.png}
\caption{Plot of first two discriminants for LDA on defense data}
\end{figure}

All in all, it shouldn't be too surprising that the LDA and MANOVA
struggled with this task: it is likely that there are higher level
clusters of teams that play similarly. LDA and MANOVA assume that all of
the classes are different, so they struggle to detect any latent
variables.

\hypertarget{conclusions-and-discussion}{%
\section{Conclusions and Discussion}\label{conclusions-and-discussion}}

In conclusion, we find that movement data can characterize player
styles. The biggest differences in play style that highlighted by these
models is playing on the outside of of three point arc versus playing on
the inside of the arc. Player locations are, in some sense, the rawest
form of data available on the NBA, less abstracted than points scored,
rebounds, or blocks; thus we are able to trust our conclusions from this
analysis more than an analysis that was based on standard statistics
like points.

We also find that it is hard to discriminate between teams using
movement data. This could be because all teams essentially stand in the
same positions on the court for the same number of time, or it could be
because there are higher level clusters of how teams play that confound
the analysis. The second option seems especially likely given that many
coaches in the NBA start out working as assistants for other coaches.

The models used in our analyses are not very complex, and as a result,
the conclusions we can make are not very deep. Getting such interesting
and interpretable results with simple methods like factor analysis is
very encouraging for further analyses of these movement data; with more
flexible models, such as a neural network, even deeper conclusions could
possibly be drawn.

\hypertarget{points-for-further-analysis}{%
\section{Points for further
analysis}\label{points-for-further-analysis}}

Although looking at where players were at a single point in time told us
a lot about NBA games, looking at their transitions could potentially
tell us even more. This type of data, which would be very high
dimensional, would not be a good candidate for factor analysis or PCA,
since the number of features would outstrip the number of players or
teams. It would, however work very well for clustering analyses. If we
had time and space for another analysis, we would like to delve into

It would also be interesting for someone with more in depth knowledge of
players in the NBA to go through and inspect what factors specific
players scored high in. Would a player like Stephen Curry, the NBA's
former most valuable player, who is known for his three point shooting,
have high factor scores with respect to zones outside of the three point
arc?

\hypertarget{data-1}{%
\section{Data}\label{data-1}}

Data is available in the \texttt{data} folder of the following github
repository: \url{https://github.com/samghelms/nba_movement_analysis}.

\end{document}
